\documentclass[a4paper,10pt]{article}
\usepackage{eumat}

\begin{document}
\begin{eulernotebook}
\eulersubheading{}
\eulerheading{BAB 4. INTERGAL TAK TENTU (Antiderivatif)}
\begin{eulercomment}
MATERI INTEGRAL TAK TENTU Dibuat Oleh :

NAMA  : SYIFA DESTA RUMAISHA\\
KELAS : MATEMATIKA E\\
NIM   : 22305141020

\end{eulercomment}
\eulersubheading{}
\begin{eulercomment}
CAKUPAN MATERI MELIPUTI DIANTARANYA:\\
-Defini Integral tak tentu\\
-Sifat- sifat integral tak tentu\\
-Integral tak tentu fungsi aljabar, trigonometri, eksponensial,
logaritma, dan komposisi fungsi\\
-Visualisasi dan kurva fungsi

1. Definisi Intergal Tak Tentu

\end{eulercomment}
\begin{eulerttcomment}
    Integral tak tentu (indefinite integral) adalah integral yang
\end{eulerttcomment}
\begin{eulercomment}
tidak memiliki batas-batas nilai tertentu, sehingga hanya diperoleh
fungsi umumnya saja disertai suatu konstanta C.

\end{eulercomment}
\begin{eulerttcomment}
    Misalkan diketahui suatu fungsi F(x) yang merupakan fungsi umum
\end{eulerttcomment}
\begin{eulercomment}
yang bersifat F'(x)=f(x), maka integral tak tentu merupakan himpunan
anti turunan F(x) dari f(x) pada interval negatif tak hingga sampai
tak hingga yang dinotasikan :

\end{eulercomment}
\begin{eulerformula}
\[
 F(x) = \int f(x)\ dx + C
\]
\end{eulerformula}
\begin{eulerprompt}
>$F(x)=('integrate(f(x),x)+c)
\end{eulerprompt}
\begin{eulerformula}
\[
F\left(x\right)=\int {f\left(x\right)}{\;dx}+c
\]
\end{eulerformula}
\begin{eulercomment}
Definisi kurva fungsi antiderifatif

\end{eulercomment}
\begin{eulerttcomment}
      Kurva fungsi antiderivatif adalah kurva yang menggambarkan
\end{eulerttcomment}
\begin{eulercomment}
hubungan antara suatu fungsi dan antiderivatifnya. Antiderivatif, juga
dikenal sebagai integral tak tentu. Dalam integral, fungsi
antiderivatif dapat dianggap sebagai "anti turunan" dari fungsi
aslinya.

Contoh :

\end{eulercomment}
\begin{eulerformula}
\[
\int 3x^2 dx
\]
\end{eulerformula}
\begin{eulerprompt}
>$F(x)=('integrate(3*x^2,x)+c)
\end{eulerprompt}
\begin{eulerformula}
\[
F\left(x\right)=3\,\int {x^2}{\;dx}+c
\]
\end{eulerformula}
\begin{eulerprompt}
>$showev('integrate(3*x^2,x)+c)
\end{eulerprompt}
\begin{eulerformula}
\[
3\,\int {x^2}{\;dx}+c=x^3+c
\]
\end{eulerformula}
\begin{eulerprompt}
>plot2d(["3*x^2","x^3","x^3+1","x^3+2","x^3+3"]): //grafik fungsinya, hasil integral, penambahan sebarang konstanta dari hasil integral 
\end{eulerprompt}
\eulerimg{24}{images/Syifa Desta Rumaisha aplikom integral tak tentu-006.png}
\begin{eulercomment}
Penyelesaiannya dengan memasukan sebarang nilai C

Contoh :\\
\end{eulercomment}
\begin{eulerformula}
\[
\int x^5 dx
\]
\end{eulerformula}
\begin{eulerprompt}
>$F(x)=('integrate(x^5,x) +c)
\end{eulerprompt}
\begin{eulerformula}
\[
F\left(x\right)=\int {x^5}{\;dx}+c
\]
\end{eulerformula}
\begin{eulerprompt}
>$showev('integrate(x^5,x)+c)
\end{eulerprompt}
\begin{eulerformula}
\[
\int {x^5}{\;dx}+c=\frac{x^6}{6}+c
\]
\end{eulerformula}
\begin{eulerprompt}
>plot2d(["x^5","x^6/6","(x^6/6)+1","(x^6/6)+2"]):
\end{eulerprompt}
\eulerimg{24}{images/Syifa Desta Rumaisha aplikom integral tak tentu-010.png}
\eulersubheading{}
\begin{eulercomment}
2. Sifat-sifat Integral Tak Tentu

\end{eulercomment}
\begin{eulerttcomment}
    Dalam perhitungan, integral tak tentu memiliki sifat-sifat yang
\end{eulerttcomment}
\begin{eulercomment}
dapat digunakan. Ada tiga sifat integral tak tentu yaitu sebagai
berikut:\\
\end{eulercomment}
\begin{eulerttcomment}
   a. Sifat Pangkat
\end{eulerttcomment}
\begin{eulerformula}
\[
\int x^n dx + c = \frac {x^n+1}{n+1} + c
\]
\end{eulerformula}
\begin{eulerprompt}
>$showev('integrate(x^n,x)+c)
\end{eulerprompt}
\begin{euleroutput}
  Answering "Is n equal to -1?" with "no"
\end{euleroutput}
\begin{eulerformula}
\[
\int {x^{n}}{\;dx}+c=\frac{x^{n+1}}{n+1}+c
\]
\end{eulerformula}
\begin{eulerttcomment}
 b. Penjumlahan dan Pengurangan
\end{eulerttcomment}
\begin{eulerformula}
\[
\int[f(x)\pm g(x)]dx = \int f(x)dx \pm \int g(x) dx
\]
\end{eulerformula}
\begin{eulerprompt}
>function f(x) &&= f(x)
\end{eulerprompt}
\begin{euleroutput}
  
                                   f(x)
  
\end{euleroutput}
\begin{eulerprompt}
>function g(x) &&= g(x)
\end{eulerprompt}
\begin{euleroutput}
  
                                   g(x)
  
\end{euleroutput}
\begin{eulercomment}
Penjumlahan
\end{eulercomment}
\begin{eulerprompt}
>$('integrate([f(x)+g(x)],x))=('integrate(f(x),x))+('integrate(g(x),x))
\end{eulerprompt}
\begin{eulerformula}
\[
\int {\left[ g\left(x\right)+f\left(x\right) \right] }{\;dx}=\int {  g\left(x\right)}{\;dx}+\int {f\left(x\right)}{\;dx}
\]
\end{eulerformula}
\begin{eulercomment}
Pengurangan
\end{eulercomment}
\begin{eulerprompt}
>$('integrate([f(x)+g(x)],x))=('integrate(f(x),x))-('integrate(g(x),x))
\end{eulerprompt}
\begin{eulerformula}
\[
\int {\left[ g\left(x\right)+f\left(x\right) \right] }{\;dx}=\int {  f\left(x\right)}{\;dx}-\int {g\left(x\right)}{\;dx}
\]
\end{eulerformula}
\begin{eulercomment}
c. Konstanta\\
\end{eulercomment}
\begin{eulerformula}
\[
\int k.f(x) dx = k \int f(x) dx
\]
\end{eulerformula}
\begin{eulerprompt}
>$('integrate(kf(x),x))=(k*'integrate(f(x),x))
\end{eulerprompt}
\begin{eulerformula}
\[
\int {{\it kf}\left(x\right)}{\;dx}=k\,\int {f\left(x\right)}{\;dx}
\]
\end{eulerformula}
\eulersubheading{}
\begin{eulercomment}
3. INTERGAL TAK TENTU FUNGSI ALJABAR

\end{eulercomment}
\begin{eulerttcomment}
    A. Defnisi
      Integral tak tentu fungsi aljabar merupakan sebuah operasi
\end{eulerttcomment}
\begin{eulercomment}
matematika yang menghasilkan fungsi lain yang turunan parsialnya akan
sama dengan fungsi asal. Dalam konteks fungsi aljabar, integral tak
tentu biasanya melibatkan fungsi-fungsi seperti polinomial,
eksponensial, dan trigonometri, dan menghasilkan fungsi yang mewakili
daerah di bawah kurva fungsi asal terhadap variabel independen.

\end{eulercomment}
\begin{eulerttcomment}
    B. Rumus-rumus integral fungsi aljabar
\end{eulerttcomment}
\begin{eulercomment}
Bentuk pertama\\
\end{eulercomment}
\begin{eulerformula}
\[
\int dx = x + C
\]
\end{eulerformula}
\begin{eulercomment}
Dalam bentuk pertama bukan berarti tidak ada konstanta yang terlibat
dalam integral tak tentu, tapi ada konstanta yaitu angka 1, di dalam
matematika biasanya angka 1 sebagai konstanta tidak dituliskan.

Bentuk kedua\\
\end{eulercomment}
\begin{eulerformula}
\[
\int k dx = kx + C
\]
\end{eulerformula}
\begin{eulercomment}
k merupakan konstanta yang berupa sebarang bilangan.

Bentuk ketiga\\
\end{eulercomment}
\begin{eulerformula}
\[
\int kx^n dx = \frac {k}{n+1} x^{n+1} + C
\]
\end{eulerformula}
\begin{eulercomment}
k dan n merupakan sebarang bilangan bulat, dengan k adalah konstanta
dan n adalah pangkat dari x dengan syarat n tidak sama dengan -1.

Bentuk keempat\\
\end{eulercomment}
\begin{eulerformula}
\[
\int k.f(x) dx = k \int f(x) dx
\]
\end{eulerformula}
\begin{eulerttcomment}
 Dengan k merupakan sebarang bilangan bulat.
\end{eulerttcomment}
\begin{eulercomment}
Bentuk kelima\\
\end{eulercomment}
\begin{eulerformula}
\[
\int (f(x)\pm g(x)) dx = \int f(x) dx \pm \int g(x) dx
\]
\end{eulerformula}
\begin{eulercomment}
Bentuk keenam\\
\end{eulercomment}
\begin{eulerformula}
\[
\int k(ax+b)^n dx = \frac {k}{a(n+1)} (ax+b)^{n+1} + C
\]
\end{eulerformula}
\begin{eulerttcomment}
 Dalam bentuk integral keenam hanya berlaku jika angka pada pangkat x
\end{eulerttcomment}
\begin{eulercomment}
adalah 1.


\end{eulercomment}
\begin{eulerttcomment}
    C. Contoh Soal & Kurva
\end{eulerttcomment}
\begin{eulercomment}
Soal 1\\
\end{eulercomment}
\begin{eulerformula}
\[
\int 5x^2 dx
\]
\end{eulerformula}
\begin{eulerprompt}
>$showev('integrate(5*x^2,x)+c)
\end{eulerprompt}
\begin{eulerformula}
\[
5\,\int {x^2}{\;dx}+c=\frac{5\,x^3}{3}+c
\]
\end{eulerformula}
\begin{eulerprompt}
>plot2d(["5*x^2","(5*x^3/3)","(5*x^3/3)+1","(5*x^3/3)+2"]):
\end{eulerprompt}
\eulerimg{24}{images/Syifa Desta Rumaisha aplikom integral tak tentu-026.png}
\begin{eulercomment}
Soal 2

\end{eulercomment}
\begin{eulerformula}
\[
f(x)= 4x+2, g(x)= 2x+1
\]
\end{eulerformula}
\begin{eulerprompt}
>function f(x) &&= 4*x+2
\end{eulerprompt}
\begin{euleroutput}
  
                                 4 x + 2
  
\end{euleroutput}
\begin{eulerprompt}
>function g(x) &&= 2*x+1
\end{eulerprompt}
\begin{euleroutput}
  
                                 2 x + 1
  
\end{euleroutput}
\begin{eulerprompt}
>$showev('integrate(f(x)+(g(x)),x)+c)
\end{eulerprompt}
\begin{eulerformula}
\[
\int {6\,x+3}{\;dx}+c=3\,x^2+3\,x+c
\]
\end{eulerformula}
\begin{eulerprompt}
>plot2d(["6*x+3","3*x^2+3*x","(3*x^2+3*x)+1","(3*x^2+3*x)+2"]):
\end{eulerprompt}
\eulerimg{24}{images/Syifa Desta Rumaisha aplikom integral tak tentu-029.png}
\begin{eulercomment}
Soal 3
\end{eulercomment}
\begin{eulerprompt}
>$showev('integrate(x*sqrt(x+2),x))
\end{eulerprompt}
\begin{eulerformula}
\[
\int {x\,\sqrt{x+2}}{\;dx}=\frac{2\,\left(x+2\right)^{\frac{5}{2}}  }{5}-\frac{4\,\left(x+2\right)^{\frac{3}{2}}}{3}
\]
\end{eulerformula}
\eulersubheading{}
\begin{eulercomment}
4. INTERGAL TAK TENTU FUNGSI NON ALJABAR (transenden)\\
\end{eulercomment}
\begin{eulerttcomment}
    4.1 Intergal Tak Tentu Fungsi Trigonometri
\end{eulerttcomment}
\begin{eulercomment}

\end{eulercomment}
\begin{eulerttcomment}
        A. Defnisi
    Integral tak tentu fungsi trigonometri merupakan operasi
\end{eulerttcomment}
\begin{eulercomment}
matematika yang digunakan untuk mencari fungsi asal sebelumnya
(biasanya ditambahkan dengan konstanta) yang ketika diambil turunan
akan menghasilkan fungsi trigonometri tersebut.\\
\end{eulercomment}
\begin{eulerttcomment}
    Secara umum, integral tak tentu fungsi trigonometri seperti
\end{eulerttcomment}
\begin{eulercomment}
sin(x), cos(x), atau tan(x) melibatkan berbagai rumus dan teknik
integral yang berbeda tergantung pada jenis fungsi trigonometri yang
terlibat.

\end{eulercomment}
\begin{eulerttcomment}
        B. Rumus-rumus integral fungsi trigonometri
\end{eulerttcomment}
\begin{eulercomment}

\end{eulercomment}
\eulerimg{19}{images/Syifa Desta Rumaisha aplikom integral tak tentu-031.png}
\begin{eulercomment}

\end{eulercomment}
\begin{eulerttcomment}
        C. Contoh Soal & Kurva
\end{eulerttcomment}
\begin{eulercomment}
Soal 1

\end{eulercomment}
\begin{eulerformula}
\[
\int 2 cos x dx
\]
\end{eulerformula}
\begin{eulerprompt}
>$showev('integrate(2*cos(x),x)+c)
\end{eulerprompt}
\begin{eulerformula}
\[
2\,\int {\cos x}{\;dx}+c=2\,\sin x+c
\]
\end{eulerformula}
\begin{eulerprompt}
>plot2d(["2*cos(x)","2*sin(x)","2*sin(x)+1","2*sin(x)+2"]):
\end{eulerprompt}
\eulerimg{24}{images/Syifa Desta Rumaisha aplikom integral tak tentu-034.png}
\begin{eulercomment}
Soal 2

\end{eulercomment}
\begin{eulerformula}
\[
\int cos (2x+1) dx
\]
\end{eulerformula}
\begin{eulerprompt}
>$showev('integrate(cos(2*x+1),x)+c)
\end{eulerprompt}
\begin{eulerformula}
\[
\int {\cos \left(2\,x+1\right)}{\;dx}+c=\frac{\sin \left(2\,x+1  \right)}{2}+c
\]
\end{eulerformula}
\begin{eulerprompt}
>plot2d(["cos(2*x+1)","sin(2*x+1)/2","(sin(2*x+1)/2)+1","(sin(2*x+1)/2)+2"]):
\end{eulerprompt}
\eulerimg{24}{images/Syifa Desta Rumaisha aplikom integral tak tentu-037.png}
\begin{eulerttcomment}
 4.2 Intergal Tak Tentu Fungsi Eksponensial
\end{eulerttcomment}
\begin{eulercomment}
\end{eulercomment}
\begin{eulerttcomment}
        A. Defnisi
     Integral dari fungsi eksponensial adalah operasi matematika yang
\end{eulerttcomment}
\begin{eulercomment}
digunakan untuk menemukan area di bawah kurva fungsi eksponensial
tertentu. Integral fungsi eksponensial merupakan proses untuk
menemukan fungsi yang, ketika di turunkan, akan menghasilkan fungsi
eksponensial tersebut.

\end{eulercomment}
\begin{eulerttcomment}
        B. Rumus-rumus integral fungsi eksponensial
\end{eulerttcomment}
\begin{eulercomment}
Secara umum, integral dari fungsi eksponensial e\textasciicircum{}x adalah:

\end{eulercomment}
\begin{eulerformula}
\[
\int e^x dx = e^x + c
\]
\end{eulerformula}
\begin{eulerprompt}
>$showev('integrate((E^x),x)+ c)
\end{eulerprompt}
\begin{eulerformula}
\[
\int {e^{x}}{\;dx}+c=e^{x}+c
\]
\end{eulerformula}
\begin{eulercomment}
di mana "C" adalah konstanta integrasi. Ini berarti hasil dari
integral ini adalah fungsi eksponensial e\textasciicircum{}x itu sendiri ditambah
dengan konstanta integrasi.

\end{eulercomment}
\begin{eulerttcomment}
       C. Contoh Soal & Kurva
\end{eulerttcomment}
\begin{eulercomment}

Soal 1\\
\end{eulercomment}
\begin{eulerformula}
\[
\int e^{3x} dx
\]
\end{eulerformula}
\begin{eulerprompt}
>$showev('integrate((E^x)^3,x)+c)
\end{eulerprompt}
\begin{eulerformula}
\[
\int {e^{3\,x}}{\;dx}+c=\frac{e^{3\,x}}{3}+c
\]
\end{eulerformula}
\begin{eulerprompt}
> \(\backslash\)plot2d(["(E^x)^3","((E^x)^3)/3","(((E^x)^3)/3)+7"],color=[blue,red,green]):
\end{eulerprompt}
\eulerimg{24}{images/Syifa Desta Rumaisha aplikom integral tak tentu-042.png}
\begin{eulercomment}
Soal 2\\
\end{eulercomment}
\begin{eulerformula}
\[
\int xe^{2x} dx
\]
\end{eulerformula}
\begin{eulerprompt}
>$showev('integrate(x*(E^x)^2,x)+c)
\end{eulerprompt}
\begin{eulerformula}
\[
\int {x\,e^{2\,x}}{\;dx}+c=\frac{\left(2\,x-1\right)\,e^{2\,x}}{4}+  c
\]
\end{eulerformula}
\begin{eulerprompt}
>plot2d(["x*(E^x)^2","((2*x-1)E^x^2)/4","(2*x-1)E^x^2/4 +1"]):
\end{eulerprompt}
\eulerimg{24}{images/Syifa Desta Rumaisha aplikom integral tak tentu-045.png}
\begin{eulerttcomment}
 4.3 Intergal Tak Tentu Fungsi Logaritma
        A. Defnisi
    Integral dari fungsi logaritma adalah operasi matematika yang
\end{eulerttcomment}
\begin{eulercomment}
digunakan untuk menemukan area di bawah kurva fungsi logaritma
tertentu.

\end{eulercomment}
\begin{eulerttcomment}
        B. Rumus integral fungsi logaritma
\end{eulerttcomment}
\begin{eulercomment}

\end{eulercomment}
\begin{eulerformula}
\[
\int log(x) dx = x log(x) - x + C
\]
\end{eulerformula}
\begin{eulerprompt}
>$showev('integrate(ln(x),x)+c)
\end{eulerprompt}
\begin{eulerformula}
\[
\int {\log x}{\;dx}+c=x\,\log x-x+c
\]
\end{eulerformula}
\begin{eulerttcomment}
        C. Contoh Soal & Kurva
\end{eulerttcomment}
\begin{eulercomment}
Soal 1

\end{eulercomment}
\begin{eulerformula}
\[
\int log(2x) dx
\]
\end{eulerformula}
\begin{eulerprompt}
>$showev('integrate(log(2*x),x)+c)
\end{eulerprompt}
\begin{eulerformula}
\[
\int {\log \left(2\,x\right)}{\;dx}+c=\frac{2\,x\,\log \left(2\,x  \right)-2\,x}{2}+c
\]
\end{eulerformula}
\begin{eulerprompt}
>plot2d(["log(2*x)","(2*x)log(2*x)-2*x/2 +1","(2*x)log(2*x)-2*x/2 +2"]):
\end{eulerprompt}
\eulerimg{24}{images/Syifa Desta Rumaisha aplikom integral tak tentu-050.png}
\eulersubheading{}
\begin{eulercomment}
5. INTERGAL TAK TENTU FUNGSI KOMPOSISI

\end{eulercomment}
\begin{eulerttcomment}
   A. Defnisi
    Integral tak tentu dari fungsi komposisi, juga dikenal sebagai
\end{eulerttcomment}
\begin{eulercomment}
"integral tak tentu dari substitusi," adalah teknik integral yang
digunakan untuk mengintegrasikan fungsi yang merupakan hasil dari
komposisi dua fungsi.

\end{eulercomment}
\begin{eulerttcomment}
  B. Contoh Soal dan kurva
\end{eulerttcomment}
\begin{eulercomment}

Soal 1\\
\end{eulercomment}
\begin{eulerttcomment}
   f(x)=x+1, g(x)=x+2
\end{eulerttcomment}
\begin{eulerprompt}
>function f(x) &= x+1
\end{eulerprompt}
\begin{euleroutput}
  
                                  x + 1
  
\end{euleroutput}
\begin{eulerprompt}
>function g(x) &= x+2
\end{eulerprompt}
\begin{euleroutput}
  
                                  x + 2
  
\end{euleroutput}
\begin{eulerprompt}
>$showev('integrate(f(g(x)),x)+c)
\end{eulerprompt}
\begin{eulerformula}
\[
\int {x+3}{\;dx}+c=\frac{x^2}{2}+3\,x+c
\]
\end{eulerformula}
\begin{eulerprompt}
>plot2d(["x+3","((x^2)/2)+3*x","(((x^2)/2)+3*x)+1","(((x^2)/2)+3*x)+2"]):
\end{eulerprompt}
\eulerimg{24}{images/Syifa Desta Rumaisha aplikom integral tak tentu-052.png}
\eulerheading{TERIMAKSIH}
\eulersubheading{}
\begin{eulerprompt}
> 
> 
\end{eulerprompt}
\end{eulernotebook}
\end{document}
